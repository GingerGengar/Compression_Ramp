\documentclass[a4paper, 12pt]{report}

\usepackage{amsmath}
\usepackage{esint}
\usepackage{comment}
\usepackage{amssymb}
\usepackage{commath}
\usepackage{geometry}
\usepackage{graphicx}
\usepackage{hyperref}
\usepackage{listings}
\usepackage{xcolor}
\usepackage{array}
\usepackage{float}
\usepackage[super]{nth}

\definecolor{codegreen}{rgb}{0,0.6,0}
\definecolor{codegray}{rgb}{0.5,0.5,0.5}
\definecolor{codepurple}{rgb}{0.58,0,0.82}
\definecolor{backcolour}{rgb}{0.95,0.95,0.92}

\def\t{\theta}
\def\a{\alpha}
\def\be{\beta}
\def\w{\omega}
\def\la{\lambda}
\def\g{\gamma}
\def\f{\frac}
\def\l{\left}
\def\r{\right}
\def\dst{\displaystyle}
\def\b{\bar}
\def\h{\hat}
\def\ph{\phi}
\def\d{\cdot}
\def\n{\nabla}
\def\p{\partial}
\def\na{\nabla}
%\def\lap{\mathcal{L}}
\def\size{0.70}
\def\sizem{0.28}
\def\tabsize{0.9cm}
\def\ltabsize{4.9cm}

\let\stdsection\section
\renewcommand\section{\newpage\stdsection}
\geometry{portrait, margin= 0.5in}

\begin{document}

\title{Compression Ramp Documentation}
\author{Hans C. Suganda}
\date{$25^{th}$ February 2022}
%\maketitle
\newpage

\lstset{
	columns=fullflexible,
	frame=single,
	breaklines=true,
	backgroundcolor=\color{backcolour},   
	commentstyle=\color{codegreen},
	keywordstyle=\color{magenta},
	numberstyle=\tiny\color{codegray},
	stringstyle=\color{codepurple},
	basicstyle=\ttfamily\footnotesize,
	keepspaces=true,                 
	numbersep=5pt,                  
	showspaces=false,               
	showtabs=false,                  
	tabsize=2
}

%\tableofcontents

\begin{center}

%Seperator
%Seperator
%Seperator
%Seperator
%Seperator
\section{Algorithm Plan}
\begin{comment}
\end{comment}

%Seperator
%Seperator
%Seperator
%Seperator
\subsection{Determinable Quantities}
\begin{comment}
\end{comment}
\begin{itemize}
\item Mach Number: Important because it informs the oblique shock wave angle
\item Static Pressure: It is important to know the static pressure at the start of the compression cowl and the static pressure at the end of the cowl so that we can determine stagnation pressure loss using the isentropic solutions and that is an important design performance metric
\item Static Temperature: Potentially useful when considering the inlet to the compressor and combustor
\end{itemize}

%Seperator
%Seperator
%Seperator
%Seperator
\subsection{Loop Dependent Variables}
\begin{comment}
\end{comment}
\begin{itemize}
\item Current Step Position
\item Current Step Angle
\item old Step Angle theta
\item New Step Angle Theta
\item Turn Angle of the Ramp
\item Gradient of the Ramp 
\item True Mach Number Before Shock
\item Mach Normal to Shock Before
\item Mach Nromal to Shock After
\item True Mach Number After Shock
\item dx 
\end{itemize}

%Seperator
%Seperator
%Seperator
%Seperator
\subsection{Needed Functionality}
\begin{comment}
\end{comment}
\begin{itemize}
\item Find Angle between two points 
\item From Angle determine gradient 
\item Given gradient and dl determine dx
\end{itemize}

%Seperator
%Seperator
%Seperator
%Seperator
\subsection{Looping Procedure}
\begin{comment}
\end{comment}
\begin{enumerate}
\item Have an initial position in x and y
\item Have the coordinates of the cowl
\item Loop from the initial position to the specified maximum iteration count
\item Update Certain Input Variables
\item Set the Mach number before shock as mach number after shock last iteration
\item Set the old angle theta as the new angle theta from last iteration
\item Find the angle of the line between the cowl and the current position
\item update current angle
\item Compute the Mach number normal to shock
\item Use the Explicit Equation
\item update Mach Normal to Shock Before
\item Use Shock Jump relations for normal Mach number and all relevant quantities
\item Update Mach Normal to Shock After
\item Compute the turn angle of the ramp
\item Use Explicit Equation and plug in shock angle
\item update turn angle of ramp
\item Compute new true Mach number
\item Update True Mach Number After Shock
\item Find Gradient of current ramp
\item new angle theta is the old + turn angle
\item Use Basic Trigonometry to compute current gradient of ramp
\item Update Value of the gradient of ramp
\item Compute value of dx
\item Update Value of dx
\item Take one small "step" forward assuming straight line (update current position)
\item From Gradient of the Ramp and current position,
\item increment x by dx
\item increment y by gradient x dx
\item Log position and probably more quantities of interest
\end{enumerate}

%Seperator
%Seperator
%Seperator
%Seperator
\subsection{Programming Standards}
\begin{comment}
\end{comment}
\begin{itemize}
\item Do not use printf for printing. Let us all be consistent and use iostream and their inherited class system
\item Do not edit main directly or tamper with the variable declarations. Do your individual testing in "Testing"
\item Do not use "using namespace as std" instead spell out std:: this is to avoid ambiguity in namespaces
\end{itemize}

%Seperator
%Seperator
%Seperator
%Seperator
%Seperator
\section{Programming Implementation}
\begin{comment}
\end{comment}

%Seperator
%Seperator
%Seperator
%Seperator
\subsection{Header.h}
\begin{comment}
\end{comment}
A header file is declared for this simulation. All files will use the header file shown below. The header files only contain function and variable declarations. The header files are to ensure that the compiler can compile each file independently with respect to one another. The keyword \url{extern} is used to denote that the definition of the variable in question could be found elsewhere.

\lstinputlisting[language=C++]{../Header.h}

%Seperator
%Seperator
%Seperator
%Seperator
\subsection{Var\_Setup.cpp}
\begin{comment}
\end{comment}
The file below is the definition of the various variables used in the program. Short comments show what the variables represent.

\lstinputlisting[language=C++]{../Var_Setup.cpp}

%Seperator
%Seperator
%Seperator
%Seperator
\subsection{Main.cpp}
\begin{comment}
\end{comment}
The single entry point of the program is shown below. \url{Main} calls upon a few high level functionalities such as \url{GenRamp} and \url{SetupFiles} which will be explained later. With the grouping of basic utilities together, \url{Main} can be concise and show a divison in "high-level" processes occuring throughout the program,


\lstinputlisting[language=C++]{../Main.cpp}

%Seperator
%Seperator
%Seperator
%Seperator
\subsection{Init.cpp}
\begin{comment}
\end{comment}
The file below is used to initialize the parameters in this problem. This function in truth could be modified to read the command line arguments or read from an input file.

\lstinputlisting[language=C++]{../Init.cpp}

%Seperator
%Seperator
%Seperator
%Seperator
\subsection{IO.cpp}
\begin{comment}
\end{comment}
The file handles the necessary data inputs and outputs. It is possible to use \url{printf} for variable output and using the \url{>} operator but since \url{C++} was used, the object-oriented method of logging data was used. Since the ramp geometry is $2$-dimensional, the ramp geometry could be fully described in its $x$ and $y$ coordiante. For simplicity, the $x$-coordinate and $y$-coordinate of the rampare logged in $2$ different files.
\\~\\The parameter of the problem is printed in a separate file in case it needs to be referenced at a later time.

\lstinputlisting[language=C++]{../IO.cpp}

%Seperator
%Seperator
%Seperator
%Seperator
\subsection{Gen\_Rmp\_Geo.cpp}
\begin{comment}
\end{comment}
The function \url{GenRamp} uses a single loop which indicates marching forward in distance along the ramp geometry. Note that that the Mach number and the angle of the ramp geometry are sufficient parameters to characterize position on the ramp. All quantities can be derived from just those $2$ parameters. Knowing that, the function \url{XMarch} overwrites the $2$ parameters so that the numerical simulation can "march" in space.

\lstinputlisting[language=C++]{../Gen_Rmp_Geo.cpp}

%Seperator
%Seperator
%Seperator
%Seperator
\subsection{Geometry.cpp}
\begin{comment}
\end{comment}
The functions implemented below are purely kinematic. They are based on basic trigonometry and basic algebraic manipulations of differential distances. The function \url{Pnts2Ang} is used when determining the wave angle of the oblique shock meanwhile the functions \url{Ang2Grad} and \url{dxComp} are used to determine the step in distance $x$ such that the length of the ramp $dl$ is preserved.

\lstinputlisting[language=C++]{../Geometry.cpp}


%Seperator
%Seperator
%Seperator
%Seperator
\subsection{Shock.cpp}
\begin{comment}
\end{comment}
The file below is the programming implementation of the various isentropic flow relations. 

\lstinputlisting[language=C++]{../Shock.cpp}


%Seperator
%Seperator
%Seperator
%Seperator
%Seperator
\section{Build Tools}
\begin{comment}
\end{comment}
The \url{Makefile} utility is used to build the binary of this project. The various functions are compiled independently of each other into object files. Then, they are compiled together to form the final binary. The script that automates this process is shown below. More complex cross-platform build tools might be available such as \url{CMake}. It seems that this project may be buildable on a unix-like system for now.

\lstinputlisting[language=bash]{../Makefile}


%Seperator
%Seperator
%Seperator
%Seperator
%Seperator
\section{Post-Processing Utilities}
\begin{comment}
\end{comment}
Visualization of the ramp geometry is done in python. The python script that shows the ramp geometry is shown below,

\lstinputlisting[language=Python]{../Ramp_Visual.py}


%Seperator
%Seperator
%Seperator
%Seperator
%Seperator
\end{center}

\end{document}


